\documentclass[10pt,a4paper,fleqn]{article}
\usepackage[utf8]{inputenc}
\usepackage{amsmath}
\usepackage{amsfonts}
\usepackage{amssymb}
\usepackage{graphicx}
\usepackage[margin=0.7in]{geometry}
\usepackage{enumitem}

\author{Andrea Franchini}
\title{Formule Utili di Analisi II}

\begin{document}
    \section*{Serie di potenze come funzioni}
    \renewcommand{\arraystretch}{2}
    \begin{tabular}{|c|l|c|}
        \hline
        $\sin x$
        &$\sum^{\infty}_{k=0}(-1)^k\frac{x^{2k+1}}{(2k+1)!}$
        &$(-\infty, +\infty)$\\
        \hline
        $\cos x$
        &$\sum^{\infty}_{k=0}(-1)^k\frac{x^{2k}}{(2k)!}$
        &$(-\infty, +\infty)$\\
        \hline
        $e^x$
        &$\sum^{\infty}_{k=0}\frac{x^k}{k!}$
        &$(-\infty, +\infty)$\\
        \hline
        $\frac{1}{1-x}$
        &$\sum^{\infty}_{k=0}x^k$
        &$(-\infty, \infty)$\\
        \hline
    \end{tabular}
    \begin{tabular}{|c|l|c|}
        \hline
        $\frac{1}{1+x}$
        &$\sum^{\infty}_{k=0} (-1)^k x^k$
        &$(-\infty, +\infty)$\\
        \hline
        $\frac{1}{1-x^2}$
        &$\sum^{\infty}_{k=0} (-1)^k x^{2k}$
        &$(-\infty, +\infty)$\\
        \hline
        $\arctan x$
        &$\sum^{\infty}_{k=0} (-1)^k\frac{x^{2k+1}}{2k+1}$
        &$[-\infty, +\infty]$\\
        \hline
        $\log (1+x)$
        &$\sum^{\infty}_{k=0}(-1)^k \frac{x^{k+1}}{k+1}$
        &$(-\infty, \infty)$\\
        \hline
    \end{tabular}

    \subsection*{Serie di Fourier}

    Consideriamo una funzione periodica $f$ di periodo $L$:
    \[
    f(x) \approx \frac{a_0}{2} + \sum_{k=1}^{\infty}\left(a_k \cos \left(\frac{2k\pi}{L}x\right) + b_k \sin\left(\frac{2k\pi}{L}x\right)\right)
    \]
    allora
    \[
    a_k = \frac{2}{L}\int^{\frac{L}{2}}_{-\frac{L}{2}}f(x)\cos\left(\frac{2k\pi}{L}x\right)dx\\
    b_k = \frac{2}{L}\int^{\frac{L}{2}}_{-\frac{L}{2}}f(x)\sin\left(\frac{2k\pi}{L}x\right)dx
    \]

    \subsection*{Uguaglianza di Parseval}

    \[
    \int^{\frac{L}{2}}_{-\frac{L}{2}}f(x)^2dx = \frac{L}{2}\left[\frac{a_0^2}{2} + \sum_{k=1}^{\infty}\left(a^2_k + b^2_k\right)\right]
    \]

    \subsection*{Ricerca di un integrale particolare\\dell'equazione $y''(t)+ay'(t)+by(t)=0$}

    \begin{itemize}
        \item Sia $f(t)=P_m(t)$, polinomio di grado $m \geq 0$, allora $y(t)=P_q(t)$ ove $q=m+r$, essendo r l'ordine minimo di derivazione con cui appare la $y(t)$ nell’equazione differenziale.
        \item Sia $f(t)=he^{kt}, k \not = 0$
        \begin{itemize}
            \item se $k$ non è radice dell'equazione caratteristica $y(t)=Ae^{kt}$
            \item se $k$ è radice $r$-pla dell'equazione caratteristica $y(t)=At^re^{kt}$
        \end{itemize}
        \item Sia $f(t)=P_m(t)e^{kt}$ allora $y(t)=P_q(t)e^{kt}$ ove
        \begin{itemize}
            \item $q=m$ se $k$ non è radice dell'equazione caratteristica
            \item $q=m+r$ se $k$ è radice $r$-pla dell'equazione caratteristica
        \end{itemize}
        \item Sia $f(t)=h\sin(kt)$ oppure $f(t)=h\cos(kt)$:
        \begin{itemize}
            \item se $\pm ik$ non sono radici dell'equazione caratteristica allora $y(t)=A\sin(kt)+B\cos(kt)$
            \item se $\pm ik$ sono radici dell'equazione caratteristica $y(t)=t(A\sin(kt)+B\cos(kt))$
        \end{itemize}
        \item Sia $f(t)=e^{pt} \cos(qt)$ oppure $f(t)=e^{pt} \sin(qt)$
        \begin{itemize}
            \item se $p \pm iq$ non sono radici dell'equazione caratteristica $y(t)=e^{pt} (A \cos(qt) + B \sin(qt))$
            \item se $p \pm iq$ sono radici dell'equazione caratteristica
            $y(t)=e^{pt} (A \cos(qt) + B \sin(qt))$
        \end{itemize}
    \end{itemize}
    \pagebreak

    \subsection*{Equazione di Eulero}

    \[
        x^2 y''(x)+ bxy'(x) + cy(x) = f(x)
    \]
    Supponiamo $x>0$ e poniamo $x=e^t$,
    \begin{multline*}
        y(x)=y(e^t)=u(t)\\
        u'(t)=y'(e^t)e^t\\
        u''(t)=y''(e^t)e^{2t}+y'(e^t)e^t\\
        y'(e^t)=u'(t)e^{-t}\\
        y''(e^t)e^{2t}=u''(t)-y'(e^t)e^t=u''(t)-u'(t)e^{-t}e^t\\
        y''(e^t)=\left[u''(t)-u'(t)\right]e^{-2t}\\
    \end{multline*}
    Quindi l'equazione di partenza diventa
    \begin{multline*}
        e^{2t}\left[u''(t)-u'(t)\right]e^{-2t}+be^tu'(t)e^{-t}+cu(t)=f(e^t)\\
        u''(t)+(b-1)u'(t)+cu(t)=f(t)\\
    \end{multline*}
    \subsubsection*{Esempi}
    \begin{minipage}{0.5\textwidth} 
        \begin{multline*}
            x^2 y''(x) + xy'(x)-y(x)=1 \quad\text{diventa}\\
            u''(t)-u(t)=1\\
            u(t)=Ae^t+Be^{-t}-1\\
            y(x)=Ax+Bx^{-1}-1\\
        \end{multline*}
    \end{minipage}
    \begin{minipage}{0.5\textwidth} 
        \begin{multline*}
            x^2 y''(x) -5xy'(x) + 8y(x)=x^3 \quad\text{diventa}\\
            u''(t)-6u'(t) + 8u(t)= e^{3t}\\
            u(t)=Ae^{4t}+Be^{2t}-e^{3t}\\
            y(x)=Ax^4+Bx^2-x^3\\
        \end{multline*}
    \end{minipage}

\end{document}