\documentclass[10pt,a4paper]{article}
% Use Helvetica
\usepackage[scaled]{helvet}
% Use it as default
\renewcommand\familydefault{\sfdefault} 
\usepackage[T1]{fontenc}\usepackage[utf8]{inputenc}

\usepackage{graphicx}
% Add margins
\usepackage[margin=0.4in]{geometry}

% Code
\newcommand{\code}{\texttt}

% Lists have tiny bullet
\renewcommand\labelitemi{\tiny$\bullet$}

\author{Andrea Franchini}
\title{Appunti di Algoritmi}

\begin{document}
\section{Complessit\`a del calcolo}
   \subsection*{Caso pessimo}
   $T_M(n) = \max\left\{T_M(x), |x|=n\right\}$\\
   $S_M(n) = \max\left\{T_M(x), |x|=n\right\}$
   \subsection*{Notazioni}
   \begin{itemize}
       \item O-grande: limite asintotico superiore.\\
       Data $g(n)$, $O(g(n)) = \left\{f(n) \mid \exists c,n_0 \left(c,n_0 > 0 : \forall n \ge n_0 0 \le f(n) \le cg(n)\right)\right\}$

       \item $\Omega$-grande: limite asintotico inferiore.\\
       Data $g(n)$, $\Omega(g(n)) = \left\{f(n) \mid \exists c,n_0 \left(c,n_0 > 0 : \forall n \ge n_0 0 \le cg(n)  \le f(n) \right)\right\}$

       \item $\Theta$-grande: limite asintotico sia superiore sia inferiore.\\
       Data $g(n)$, $\Theta(g(n)) = \left\{f(n) \mid \exists c_1,c_2,n_0 \left(c_1,c_2,n_0 > 0 : \forall n \ge n_0 0 \le c_1g(n) \le f(n) \le c_2g(n)\right)\right\}$
   \end{itemize}
\section{Teoremi di accelererazione lineare}
\begin{itemize}
    \item Se $L$ \`e accettato da una MT $M$ a $k$ nastri con complessit\`a $S_M(n)$, per ogni $c > 0 (c \in R)$ si pu\`o costruire una MT $M'$ a $k$ nastri con complessit\`a $S_{M'}(n) < c S_M(n)$
    
    \item Se $L$ \`e accettato da una MT $M$ a $k$ nastri con complessit\`a $S_M(n)$, si pu\`o costruire una MT $M'$ a 1 nastro (\textit{non} a nastro singolo) con complessit\`a $S_{M'}(n) = S_M(n)$

    \item Se $L$ \`e accettato da una MT $M$ a $k$ nastri con complessit\`a $S_M(n)$, per ogni $c > 0 (c \in R)$ si pu\`o costruire una MT $M'$ a 1 nastro con complessit\`a $S_{M'}(n) < cS_M(n)$

    \item Se $L$ \`e accettato da una MT $M$ a $k$ nastri con complessit\`a $T_M(n)$, per ogni $c > 0 (c \in R)$ si pu\`o costruire una MT $M'$ (a $k+1$ nastri) con complessit\`a $T_{M'}(n) = \max \left\{n+1, cT_M(n)\right\}$
\end{itemize}
\paragraph{Conseguenze pratiche}
\begin{itemize}
    \item Lo schema di dimostrazione \`e valido per qualsiasi tipo di modello di calcolo, quindi anche per calcolatori reali (es.: aumentare il parallelismo fisico (16bit $\rightarrow$ 32bit $\rightarrow \dots$)).
    \item Aumentando la potenza di calcolo in termini di risorse disponibili si pu\`o aumentare la velocit\`a di esecuzione, ma il miglioramento \`e al pi\`u lineare.
    \item Miglioramenti di grandezza superiore possono essere ottenuti solo cambiando algoritmo e non in modo automatico.
\end{itemize}
\section{Macchina RAM}
\begin{tabular}{l|l|l}
    \hline
    Comando & Operazione & Complessit\`a\\
    \hline
    \code{LOAD X} & \code{M[0] = M[X]} & \\
    \code{LOAD= X} & \code{M[0] = X} & \\
    \code{LOAD* X} & \code{M[0] = M[M[X]]} & \\

\end{tabular}

\end{document}