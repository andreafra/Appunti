\newglossaryentry{imprenditore}{
    name=imprenditore,
    description={chi esercita professionalmente un'attività economica
    organizzata al fine della produzione o dello scambio di beni o di servizi}
}

\newglossaryentry{impresa}{
    name=impresa,
    description={attività economica organizzata, svolta professionalmente, al fine della
    produzione o dello scambio di beni o di servizi}
}

\newglossaryentry{lavsub}{
    name=lavoratore subordinato,
    description={chi si obbliga, mediante retribuzione a
    collaborare nell'\gls{impresa}, prestando il proprio lavoro, intellettuale o
    manuale, alle dipendenze e sotto la direzione dell'\gls{imprenditore}}
}

\newglossaryentry{societa}{
    name=società,
    description={contratto con cui \emph{due o più persone} conferiscono beni
    o servizi per l’esercizio in comune di un’attività economica allo scopo di
    dividerne gli \glspl{utile}}
}

\newglossaryentry{azienda}{
    name=azienda,
    description={\emph{complesso dei beni organizzati} dall'\gls{imprenditore} per
    l'esercizio dell'\gls{impresa}}
}

\newglossaryentry{ditta}{
    name=ditta,
    description={\emph{nome commerciale} scelto dall’imprenditore per
    esercitare l’\gls{impresa}: è un segno distintivo che consente ai consumatori di
    identificare l’impresa, ha valore commerciale e pertanto la legge ne garantisce
    l'\emph{uso esclusivo}}
}

\newglossaryentry{utile}{
    name=utile,
    description={indica la differenza tra ricavi e costi di un'impresa. Se tale
    differenza è positiva viene comunemente chiamato \emph{profitto}, in caso
    contrario viene chiamato perdita},
    plural=utili
}

\newglossaryentry{shareholder}{
    name=shareholder,
    description={I proprietari dell'impresa}
}

\newglossaryentry{stakeholder}{
    name=stakeholder,
    description={Insieme delle parti interessate (management, finanziatori a titolo oneroso,
    fornitori, clienti, dipendenti, organizzazioni sindacali, concorrenti, Stato}
}

\newglossaryentry{rischio}{
    name=rischio,
    description={impossibilità di prevedere con certezza gli esiti futuri delle 
    decisioni in merito alle attività dell’impresa (``probabilità di un evento
    e delle sue conseguenze'')}
}

\newglossaryentry{bizmodel}{
    name=business model,
    description={Il piano dell'impresa per creare, distrubuire e raccogliere
    valore}
}

\newglossaryentry{bizplan}{
    name=business plan,
    description={Il business plan è la descrizione dell’idea imprenditoriale in
    cui si dimostra che l’attività proposta merita fiducia più di altre
    possibilità di investimento}
}

\newglossaryentry{personagiuri}{
    name=personalità giuridica,
    description={un soggetto giuridico cui fanno capo diritti e doveri}
}

\newglossaryentry{formagiuri}{
    name=forma giuridica,
    description={la tipologia giuridica del soggetto a cui fa
    capo l’attività e le norme ad essa conseguenti}
}

\newglossaryentry{respill}{
    name=responsabilità illimitata,
    description={l’imprenditore (i soci) risponde (rispondono)
    delle perdite dell’impresa con tutto il suo (loro) patrimonio. Ad esempio,
    per pagare gli stipendi ai lavoratori l’imprenditore può essere
    costretto dal curatore fallimentare a vendere la propria abitazione}
}

\newglossaryentry{resplim}{
    name=responsabilità limitata,
    description={i soci rispondono delle perdite dell’impresa con i capitali
    conferiti nell’impresa. Il patrimonio personale dei soci (immobili, conti
    correnti bancari a loro intestati) non è intaccato dalle perdite dell’impresa}
}

\newglossaryentry{piccoloimpr}{
    name=piccolo imprenditore,
    description={``sono piccoli imprenditori coltivatori diretti del
    fondo, gli artigiani, i piccoli commercianti, coloro che esercitano un’attività
    professionale organizzata prevalentemente con il lavoro proprio e dei
    componenti della famiglia''}
}

\newglossaryentry{contabilita}{
    name=contabilità,
    description={Processo di individuazione, misurazione, analisi, interpretazione,
    comunicazione di informazioni che consentono di esprimere giudizi e valutazioni
    economiche sull’impresa. Sistema di rilevazione continuadi qualunque evento di
    rilevanza economico-finanziaria dell’impresa}
}

\newglossaryentry{esercizio}{
    name=esercizio,
    description={Periodo di tempo, generalmente 1/1 – 31/12, ma a seconda dei settori
    produttivi e/o di particolari esigenze l’inizio e la fine dell’esercizio possono
    essere diverse, comunque di durata 12 mesi}
}

\newglossaryentry{bilancio}{
    name=bilancio,
    description={È un documento redatto con la finalità di informare i diversi \glspl{stakeholder}
    sulla situazione economica, finanziaria e patrimoniale dell’impresa in un determinato
    \gls{esercizio}}
}

\newglossaryentry{ifrsias}{
    name=IFRS/IAS,
    description={(International Financial Reporting Standards/International Accounting Standards)}
}

\newglossaryentry{sp}{
    name=stato patrimoniale,
    description={Documento del \gls{bilancio} che descrive la situazione patrimoniale dell’impresa in un determinato istante, normalmente il 31/12 di ciascun anno}
}

\newglossaryentry{ce}{
    name=conto economico,
    description={Documento del \gls{bilancio} che riassume i flussi di ricavi e costi avvenuti nell’esercizio}
}

\newglossaryentry{rf}{
    name=rendiconto finanziario,
    description={(o schema di cash-flow) Documento del \gls{bilancio} che presenta i flussi di cassa che hanno interessato l’impresa nell’esercizio}
}

\newglossaryentry{attivita}{
    name=attività,
    description={Risorsa controllata dall’impresa, risultato di operazioni svolte
    in passato, dalla quale ci si attende un afflusso di \emph{benefici economici futuri}}
}

\newglossaryentry{passivita}{
    name=passività,
    description={Obbligazioni assunte dall’impresa in relazione ad operazioni e
    altri fatti verificatisi in passato, ossia \emph{impegni irrevocabili} a tenere un
    certo comportamento per effetto di disposizioni contrattuali, di leggi o di prassi consolidate}
}

\newglossaryentry{patrnet}{
    name=patrimonio netto,
    description={\emph{Valore residuo} delle attività dell’impresa dopo aver dedotto tutte le passività}
}

\newglossaryentry{fairvalue}{
    name=fair value,
    description={Corrispettivo al quale un'attività può essere scambiata, o una
    passività estinta, tra parti consapevoli e disponibili, in una transazione tra parti terze e indipendenti}
}

\newglossaryentry{imptest}{
    name=impairment test,
    description={Verifica che le attività in bilancio siano iscritte ad un valore non superiore a quello effettivamente recuperabile}
}

\newglossaryentry{costo}{
    name=costo,
    description={In \textbf{CE}: Costo di acquisto, stimabile col metodo FIFO (First In - First Out)
    o del costo medio ponderato.\newline
    In \textbf{CI}: il valore, espresso in termini monetari, del consumo delle risorse
    impiegate per il raggiungimento di un obiettivo prefissato
    (quale la realizzazione di un prodotto, l’erogazione di un servizio, il
    funzionamento di un’unità organizzativa ...)}
}
\newglossaryentry{valrealizzo}{
    name=valore di realizzo,
    description={Prezzo medio di vendita stimato}
}

\newglossaryentry{rateo}{
    name=rateo,
    description={Evento economico che precede evento finanziario}
}

\newglossaryentry{risconto}{
    name=risconto,
    description={Evento finanziario precede evento economico}
}

\newglossaryentry{patrnetto}{
    name=patrimonio netto,
    description={Valore dei diritti vantati sull’impresa dagli azionisti per il
    capitale versato e/o maturati in seguito alle attività di funzionamento dell’impresa}
}

\newglossaryentry{TFR}{
    name=TFR,
    description={Trattamento di Fine Rapporto}
}

\newglossaryentry{contoeconomico}{
    name=conto economico,
    description={Documento di bilancio che presenta i \emph{flussi economici in entrata ed uscita} dall’impresa nel corso
    dell’esercizio contabile, determina l’\emph{utile di esercizio} dell’impresa come differenza tra i costi e i
    ricavi dell’esercizio e mostra se e quanto l’impresa \emph{remunera il capitale investito}}
}

\newglossaryentry{ammortamento}{
    name=ammortamento,
    description={Valore della ``quota'' delle risorse di utilità pluriennale
    (attività non correnti) che viene ``consumata'' dalla produzione o ``deperisce''
    per obsolescenza tecnologica}
}

\newglossaryentry{ebit}{
    name=EBIT,
    description={Earnings Before Interest and Taxes}
}

\newglossaryentry{vitautile}{
    name=vita utile,
    description={stima del periodo (anni) in cui un
    bene verrà utilizzato dall’impresa}
}

\newglossaryentry{CPI}{
    name=CPI,
    description={costo pieno industriale, cioè la somma dei costi di prodotto.}
}

\newglossaryentry{investimento}{
    name=investimento,
    description={progetto che, a fronte di un’immobilizzazione iniziale di risorse, genera reddito (e
    conseguenti flussi di cassa) nel futuro, tale da remunerare le risorse investite in misura sufficiente a
    giustificarne il rischio}
}

\newglossaryentry{rendinvest}{
    name=rendimento di un investimento,
    description={valore creato dal progetto di investimento per gli investitori}
}

\newglossaryentry{riscinvest}{
    name=rischio di un investimento,
    description={incertezza sugli esisti dell’investimento}
}

\newglossaryentry{VAN}{
    name=VAN,
    description={Valore Attuale Netto, metodo per valutare un investimento}
}

\newglossaryentry{NCF}{
    name=NCF,
    description={Net Cash Flow}
}

\newglossaryentry{budget}{
    name=budget,
    description={il budget è un programma di azione espresso in termini
    quantitativi (monetari), e che copre un predefinito arco
    temporale, solitamente pari a un esercizio}
}

\newglossaryentry{md}{
	name={materiali diretti},
	description={Materie prime, componenti, semilavorati associabili direttamente alla pro-
		duzione di un determinato prodotto/servizio}
}

\newglossaryentry{ld}{
	name={lavoro diretto},
	description={Il lavoro degli addetti alle operazioni di trasformazione fisica degli input e di assemblaggio dei componenti}
}

\newglossaryentry{VA}{
	name={valore attuale},
	description={il valore di un dato oggetto al tempo presente}
}

\newglossaryentry{PI}{
	name={profitability index},
	description={è una misura dell’intensità di creazione di valore}
}

\newglossaryentry{MON}{
	name={MON},
	description={Margine Operativo Netto, cio\`e MON = MOL - accantonamenti - ammortamenti }
}

\newglossaryentry{costopport}{
	name={costi
opportunità},
	description={misura del reddito potenziale al quale si rinuncia quando una
		determinata scelta implica l’esclusione di un corso d’azione alternativo}
}