% !TeX root = ../economia.tex
\chapter{Concorrenza}

\section{Massimizzazione del profitto}
Decisione fondamentale per le imprese: definire la quantità $q$ di un bene da produrre per
massimizzare il profitto.
\[
\max_q \pi = RT(q) - CT(q) \\ \frac{\delta \pi}{\delta q} = RM(q) - CM(q) = 0 \\ RM(q) = CM(q)
\]
dove $RM(q)$ è il \emph{ricavo marginale}, cioè la derivata prima del \emph{ricavo}, e $CM(q)$ è il \emph{costo marginale}, cioè la derivata prima del \emph{costo}, ovvero una variazione nel \emph{costo totale} che deriva dalla
produzione di un'unità aggiuntiva in output.

\subsection{Costi}

\paragraph{Costo medio fisso}
rapporto tra i costi fossi e la quantità di output prodotta.
\[
AFC(q) = \frac{CF}{q}
\]

\paragraph{Costo medio variabile}
rapporto tra i costi variabili e la quantità di output prodotta.
\[
AVC(q) = \frac{CV}{q}
\]

\paragraph{Costo medio totale}
rapporto tra i costi totali e la quantità di output prodotta.
\[
ATC(q) = AFC(q) + AVC(q) = \frac{CF + CV(q)}{q}
\]

\subsection{Determinazione del prezzo di mercato}
\begin{itemize}
	\item Se $p$ > prezzo di equilibrio si ha un \emph{eccesso di offerta}: alcuni produttori non riescono a vendere; il
	prezzo si riduce per vendere ai consumatori con prezzo di riserva più basso.
	\item Se $p$ < prezzo di equilibrio si ha un \emph{eccesso di domanda}: alcuni consumatori sarebbero disposti a comprare il bene ma questo non è
disponibile; la quantità offerta e il prezzo aumentano.
\end{itemize}

\subsection{Forme di mercato}
La capacità delle imprese di massimizzare il profitto dipende da diversi fattori che determinano la \emph{forma di mercato}:
\begin{itemize}
	\item Numero di concorrenti (imprese che producono beni che i consumatori percepiscono come stretti
sostituti)
	\item Natura del prodotto (differenziazione prodotto rispetto ai concorrenti)
	\item Grado di libertà di entrata (o uscita) delle imprese nel mercato
	\item Quantità di informazione detenuta da imprese e consumatori
	\item Grado di controllo sul prezzo da parte delle imprese
\end{itemize}

I mercati reali si collocano in un continuum tra concorrenza perfetta e monopolio:
\paragraph{Concorrenza perfetta} infinite imprese nell’industria, massimo livello di competizione.
\paragraph{Monopolio} una sola impresa nell’industria, minimo livello di competizione.
\paragraph{Concorrenza monopolistica o imperfetta} molte imprese, prodotto differenziato.
\paragraph{Oligopolio} poche imprese, moltitudine di clienti, barriere all’ingresso medio-alte.
\paragraph{Monopolio bilaterale} presenza di un solo soggetto offerente dal lato dell'offerta ed un solo soggetto
acquirente dal lato della domanda.

\section{Concorrenza perfetta}

Il modello della concorrenza perfetta si basa su quattro ipotesi fondamentali:
\begin{enumerate}
	\item Esiste un numero molto elevato di imprese nel mercato; la singola impresa produce una quota
	trascurabile dell’offerta totale
	\item Tutte le imprese producono un prodotto identico; in altre parole, il prodotto è omogeneo
	\item Acquirenti e venditori hanno una conoscenza perfetta del mercato
	\item Esiste completa libertà di entrata e di uscita da parte di nuove imprese
	\item Le imprese utilizzano la medesima tecnologia produttiva
\end{enumerate}

La concorrenza perfetta è una forma di mercato \emph{estrema}, infatti le imprese \emph{non hanno alcun potere di influenzare} il prezzo del prodotto e il prezzo a cui vendono è determinato dall’interazione della domanda e dell’offerta complessiva di
mercato.

In altri termini, le imprese sono price-taker: se fissassero un prezzo superiore a quello di mercato, non venderebbero nulla, mentre se fissassero un prezzo inferiore a quello di mercato, non avrebbero la capacità di soddisfare
l’intero mercato.

Non esistono
posizioni di privilegio determinate dal \emph{know-how} o rendite esclusive derivanti da \emph{brevetti}. La stessa
tecnologia implica la medesima curva dei costi di produzione per ogni impresa:
\begin{itemize}
	\item Nel \emph{breve periodo} possono, comunque, esserci lievi differenze nella struttura di costo delle
	imprese concorrenti, queste differenze consentono di distinguere le imprese, in imprese
	marginali e imprese con ``extra-profitto''
	\item Nel \emph{lungo periodo}, invece la struttura dei costi è la stessa in ogni impresa
\end{itemize}

\subsection{Curva di offerta individuale}
La \emph{curva di offerta individuale} esprime, per ogni livello del prezzo, la quantità ottimale $q$ di produzione
del bene, cioè quella che consente all’impresa di massimizzare il profitto $p$.
Essendo l’impresa price-taker, $p$ non dipende dalla quantità prodotta dalla singola impresa $q$:
\[
RT(q) = pq \\ RM(q) = p
\]

\paragraph{Condizione di massimizzazione del profitto} $RM(q) = CM(q)$, cioè:
\[
p = CM(q)
\]

\paragraph{Condizione minima di produzione} $\pi = pq - CT(q) > 0$. Si ottiene quindi che il prezzo deve essere superiore al costo medio affinché l’impresa sia in grado
di ottenere profitti positivi:
\[
p > \frac{CT(q)}{q}
\]

\subsection{Effetti nel lungo periodo}
Nel lungo periodo, se le imprese già operative ottengono profitti positivi ($p$ > costo medio totale), nuove
imprese saranno attirate nel mercato, innescando il seguente ciclo:
\begin{enumerate}
	\item nuove imprese entrano nel mercato attratte dal profitto
	\item l’offerta $\uparrow$­ e il prezzo di equilibrio $p$ $\downarrow$
	\item per alcune imprese diviene $p$ < costo medio totale (ATC)
e escono
	\item l’offerta $\downarrow$ e $p$ $\uparrow$ 
\end{enumerate}

\paragraph{Equilibrio di lungo periodo} entrata e uscita cessano quando non sono più possibili profitti. A quel punto:
\begin{itemize}
	\item Rimangono sul mercato solo le imprese più efficienti che producono al costo medio minimo
	\item Le imprese conseguono profitti nulli
	\item L’output è prodotto al costo unitario più basso possibile
	\item Al venditore è pagato solo il costo di produzione, quindi dal punto di vista dell’impresa la concorrenza
	non è desiderabile
\end{itemize}

\section{Monopolio}

A volte in un mercato c’è un’unica impresa produttrice (monopolista), perchè possono esistere ostacoli insormontabili che impediscono ad altre imprese di entrare
e competere.

Il monopolista è \emph{price-maker}: a differenza della concorrenza perfetta, fronteggia l’intera curva di domanda di mercato e il \emph{prezzo} al quale egli vende il prodotto \emph{non è indipendente} dalla quantità venduta.

\subsection{Nascita di un monopolio}
Il monopolio è generato dalla presenza di barriere all’entrata, che possono essere di tre tipi:

\subsubsection{Barriere strutturali}
Sono barriere che dipendono da caratteristiche di tecnologia e domanda e rendono l’entrata molto costosa:
\begin{itemize}
	\item Controllo esclusivo di input fondamentali (per esempio, risorse scarse)
	\item \emph{Monopolio naturale} dovuto ad economie di scala: al crescere della quantità di produzione, aumenta la
	produttività dei fattori produttivi nel breve e lungo periodo
	\item Lock-in dei consumatori (economie di rete)
\end{itemize}

\subsubsection{Barriere strategiche}
Sono barriere generate da strategie messe in atto dalle imprese già presenti sul mercato volte a \emph{inibire l’entrata di potenziali rivali}
e \emph{indurre l’uscita dei nuovi concorrenti}, una volta che l’entrata è avvenuta. Alcune strategie di esempio sono: prezzi predatori, prezzi minori del costo marginale praticati temporaneamente per
danneggiare nuovi concorrenti e indurli ad uscire dal mercato.

\subsubsection{Barriere istituzionali}
Sono barriere create dall’intervento dello Stato (monopolio legale) che impedisce l’entrata ai
concorrenti tramite \emph{brevetti}, \emph{licenze} e \emph{appalti}.
Lo Stato per assicurarsi entrate fiscali detiene il monopolio di determinati
prodotti o servizi (\emph{monopolio fiscale}).

\subsection{Ricavi del monopolista}

Essendo l’impresa \emph{price-maker}, i ricavi totali sono dati da:
\[
RT(q) = p(q) \cdot q
\]
Il ricavo marginale è quindi:
\[
RM(q) = p(q) \cdot \left( 1 - \frac{1}{\epsilon} \right)
\]

dove $\epsilon$ è l'elasticità della domanda al prezzo (in valore assoluto).

\subsubsection{Ricavi del monopolista}
La \emph{condizione di massimizzazione del profitto} è $RM(q) = CM(q)$.

Si ottiene quindi:
\[
p(q) = \frac{CM(q)}{1 - \frac{1}{\epsilon}}
\]

Il monopolista fissa un prezzo al di sopra dei costi marginali (si dice che ha potere di mercato); il potere di mercato è tanto maggiore quanto meno la domanda risponde alle variazioni del prezzo
(cioè più è rigida).

\subsubsection{Monopolio ed elasticità}
Rispetto alla concorrenza perfetta, il monopolista è in grado di ottenere profitti positivi perché ha potere
di mercato ($p>CM$).

Nel tratto in cui la domanda è elastica al prezzo, la variabile dell'elasticità è maggiore di 1. In questo
tratto della domanda, i ricavi marginali (RM) sono positivi. Il potere di mercato è inversamente
proporzionale all'elasticità: tanto minore è l’elasticità, tanto maggiore è la differenza tra il prezzo di
vendita ed i costi marginali (MC).

\subsubsection{Mark-up che massimizza il profitto}
La condizione di massimizzazione del profitto è $RM(q) = CM(q)$ da cui si ricava il \emph{mark-up del monopolista}:
\[
\frac{p - CM}{p} = \frac{1}{\epsilon}
\]

\subsubsection{Effetti nel lungo periodo}
\begin{enumerate}
	\item Il monopolista può modificare tutti i fattori produttivi, come in concorrenza perfetta
	\item Il monopolista massimizza ancora il profitto nel punto di incontro tra le curve di RM e CM di lungo
	periodo:
può quindi ottenere profitti positivi anche nel lungo periodo
	\item Quando i fattori che danno origine al monopolio sono anch’essi soggetti a pressioni concorrenziali, il
	profitto tende a ridursi
\end{enumerate}

\subsubsection{Intervento dello Stato per ridurre il potere di mercato}
Lo Stato si fa imprenditore e gestisce direttamente la produzione, trasformando dei monopoli naturali in imprese pubbliche.
Si denota una riduzione dell’efficienza produttiva e l'applicazione di logiche clientelari.

Viene applicata una \emph{regolamentazione} al mercato.

L'uso delle \emph{leggi antitrust} servono come stimolo alla concorrenza:
\begin{itemize}
	\item impediscono/approvano fusioni
	\item frazionano imprese divenute troppo grandi
	\item impediscono comportamenti volti a ridurre la concorrenza
\end{itemize}



\section{Monopolio vs Concorrenza}
Per i consumatori è un bene?
Rispetto alla concorrenza perfetta, in monopolio si produce una \emph{quantità minore} ad un \emph{prezzo
maggiore}: ciò determina la \emph{perdita netta del monopolio}, cioè una perdita di surplus del consumatore, del
quale si appropria il monopolista (generando un’inefficienza allocativa).

\paragraph{Surplus del consumatore} è dato dalla somma che un consumatore sarebbe disposto a pagare
per un certo bene meno la somma che egli effettivamente paga per quel bene.

\paragraph{Surplus del produttore} è dato dalla differenza tra la somma totale incassata dal produttore ed il
costo di produzione.

\subsection{Discriminazioni di prezzo}
Praticare il prezzo di monopolio potrebbe non essere la strategia vincente:
\begin{itemize}
	\item la domanda è solitamente formata da (gruppi di) consumatori che hanno prezzi di riserva diversi
	\item se il monopolista riesce ad inferire tali prezzi di riserva, può praticare prezzi più alti a chi è
	disposto a pagare di più ed appropriandosi di un maggiore surplus del consumatore
\end{itemize}

\subsubsection{Discriminazione di prezzo di primo tipo}
Si ha discriminazione di prezzo del primo tipo (o perfetta) quando il monopolista riesce ad
appropriarsi interamente del surplus dei consumatori, praticando a ciascuno di essi un prezzo
corrispondente al prezzo di riserva.

È la condizione ideale per il monopolista che si appropria di tutto il surplus del consumatore, ma si
tratta di un caso teorico, infatti praticarla avrebbe costi altissimi e richiederebbe di elaborare moltissime informazioni; inoltre i consumatori inoltre non avrebbero nessun incentivo a rivelare il loro vero prezzo di riserva.

\subsubsection{Discriminazione di prezzo di secondo tipo}
Si discrimina il prezzo sulla base di caratteristiche dei consumatori che non sono direttamente
osservabili \emph{inferendo il prezzo di riserva sulla base di clausole contrattuali} che regolano l’acquisto del bene.
Un esempio è una tariffa aerea scontata se il viaggiatore si trattiene nel fine settimana: considerando che viaggi di lavoro normalmente avvengono durante la settimana si discriminano i prezzi tra coloro che viaggiano per lavoro e per piacere \emph{tramite una clausola contrattuale}.

\subsubsection{Discriminazione di prezzo di terzo tipo}
Si discrimina il prezzo sulla base di caratteristiche direttamente osservabili dei consumatori.
Si praticano prezzi maggiori (minori) a gruppi di consumatori le cui caratteristiche lasciano supporre
che i loro prezzi di riserva siano più alti (o più bassi), ossia a gruppi di consumatori la cui domanda è meno (più) elastica\footnote{Per funzionare i mercati devono essere separati!}.
Ne sono un esempio il teatro a prezzi inferiori per gli studenti/pensionati
o tariffe speciali dei treni per i giovani.

\subsection{Collusione}

Talvolta i prezzi di monopolio sono praticati anche in presenza di molteplici imprese
mediante la \emph{collusione}, un accordo tra più imprese per il conseguimento di obiettivi comuni con il fine ultimo di
massimizzare il profitto.

Le imprese che colludono replicano le azioni tipiche di un monopolista, ma in un mercato in cui
operano un numero limitato di produttori e le imprese possono colludere per mantenere i prezzi
alti.